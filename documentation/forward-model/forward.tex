\documentclass[justified, nobib]{tufte-handout}
\usepackage{amsmath,amsfonts,amssymb}

% add bibliography
\usepackage{biblatex}
\addbibresource{/Users/mf/Documents/Sources/My Library.bib}

% add graphics
\usepackage{graphicx}

% add hyperlinks
\usepackage{hyperref}

% add code listings
\usepackage{listings}

\title{Fourier-space Form of the Biot-Savart Law}
\author{Mykhailo Flaks}

\begin{document}

\maketitle

\section{Introduction}
Magnetostatics deals with magnetic fields in the presence of steady currents or magnetization. The magnetic field due to such sources can be determined using the Biot-Savart law. We limit considerations to the case of steady currents. 

\section{Biot-Savart Law in Real Space}
The Biot-Savart law for a steady current distribution \( \mathbf{J}(\mathbf{r}) \) in real space is:
\[
\mathbf{B}(\mathbf{r}) = \frac{\mu_0}{4\pi} \int \frac{\mathbf{J}(\mathbf{r'}) \times (\mathbf{r} - \mathbf{r'})}{|\mathbf{r} - \mathbf{r'}|^3} \, d^3\mathbf{r'}
\]
where \(\mu_0\) is the permeability of free space. The integration is over the entire space where the current distribution is non-zero, \( \mathbf{r} \) is the position vector of the point where the magnetic field \(\mathbf{B}(\mathbf{r})\) is to be determined, and \( \mathbf{r'} \) is the position vector of the current element.

\section{Fourier Transformation}
The convention for the physical Fourier transform used is:
\begin{align*}
g(k) &= \int G(x) \exp(-ikx) \, dx \\
G(x) &= \frac{1}{2\pi} \int g(k) \exp(ikx) \, dk
\end{align*}
where \( k \) is the wave-vector. In DFT, the cyclic frequency \( f \) is used and is related to the wave-vector as:
\[
k = 2\pi f
\]

\section{Biot-Savart Law in Fourier Space}
In Fourier space, the Biot-Savart law becomes:
\[
\hat{\mathbf{B}}(k) = i\mu_0 k \times \hat{\mathbf{J}}(k)
\]

To find the magnetic field in real space, one can take the inverse Fourier transform:
\[
\mathbf{B}(\mathbf{r}) = \frac{1}{2\pi} \int e^{ik\mathbf{r}} \hat{\mathbf{B}}(k) \, dk
\]

\section{Two-dimensional Fourier Transform}

When dealing with planar distributions, it is convenient to use two-dimensional Fourier transform. Assuming distributions in the \(x-y\) plane, the Fourier transform is:

\begin{equation}
\hat{g}(k_x, k_y) = \int \int G(x, y) \exp(-i(k_x x + k_y y)) \, dx \, dy
\end{equation}


\section{Conclusion}
By using the Fourier-space representation of the Biot-Savart law and the specific conventions for the Fourier transform, we can efficiently compute the magnetic field from a given current distribution. The approach is extended for two-dimensional distributions and can be adapted for various physical situations.

\end{document}

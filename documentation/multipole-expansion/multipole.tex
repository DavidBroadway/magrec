\documentclass[justified]{tufte-handout}

% Add your own customizations, packages, and settings here

% Set the document metadata
\title{Inverse Problem of Magnetostatics}
% \author{Your Name}
% \date{\today}

\begin{document}

\maketitle

\section{Multipole expansion}

Biot---Savart law is given by

\begin{equation}
\mathbf{B}(\mathbf{r}) = \frac{\mu_0}{4\pi} \int \frac{\mathbf{J}(\mathbf{r}') \times (\mathbf{r} - \mathbf{r}')}{|\mathbf{r} - \mathbf{r}'|^3} \, d^3\mathbf{r}'
\end{equation}
and in case of the magnetisation $\mathbf{M}$ we have

\begin{equation}
\mathbf{B}(\mathbf{r}) = \frac{\mu_0}{4\pi} \int \frac{\left[\nabla\times\mathbf{M}(\mathbf{r}')\right] \times (\mathbf{r} - \mathbf{r}')}{|\mathbf{r} - \mathbf{r}'|^3} \, d^3\mathbf{r}'
\end{equation}

The question I ask myself is: are any statistical properties of the magnetic field
connected to the statistical properties of the magnetisation or current density distribution that
produces it?  

\section{Popcorn model}

The popcorn model is a model of a magnetised material that consists of a large number of 
small magnetic dipoles.  The dipoles are randomly oriented and their positions are random.
The magnetisation of the material is given by the sum of all the dipoles.  The magnetic field
is given by the sum of the magnetic fields of all the dipoles.  The magnetic field of a single
dipole is given by

\begin{equation}
\mathbf{B}(\mathbf{r}) = \frac{\mu_0}{4\pi} \frac{3\mathbf{m}(\mathbf{r}')\cdot(\mathbf{r} - \mathbf{r}')(\mathbf{r} - \mathbf{r}') - |\mathbf{r} - \mathbf{r}'|^2\mathbf{m}(\mathbf{r}')}{|\mathbf{r} - \mathbf{r}'|^5}
\end{equation}
where $\mathbf{m}(\mathbf{r}')$ is the magnetic moment of the dipole at position $\mathbf{r}'$.

\section{Reconstructing magnetic field components}

We have a set of measurements of the magnetic field component $B_{NV}(\mathbf{r}_i)$ at a set of points $\{\mathbf{r}_i\}_{i\in I}$. The first step usually assumed is that it is possible to find all other magnetic field components $B_x, B_y, B_z$. It is done in Fourier space:

\begin{equation}
\end{equation}

\section{Neural network encoding current distribution}

\cite{wang2021} 

\end{document}
